\section{Discussion}
\label{sec:discussion}

\textbf{Strengths}
\begin{itemize}
    \item \textbf{Open-vocabulary reasoning:} Capable of handling novel room and object categories without retraining \cite{ref1}.
    \item \textbf{Structured reasoning:} Using scene graphs as a bridge allows the LLM to reason effectively about geometry and topology \cite{ref2}.
    \item \textbf{Robustness:} Demonstrated resilience to segmentation errors and real-world domain shifts \cite{ref4}.
\end{itemize}

\textbf{Limitations}
\begin{itemize}
    \item \textbf{Perception dependency:} heavily relies on ground-truth or high-quality semantics; limited by "garbage-in/garbage-out" \cite{ref5}.
    \item \textbf{Latency:} LLM queries dominate the runtime, and full graph re-computation scales poorly \cite{ref9}.
    \item \textbf{Cost blindness:} High-level planning decouples reasoning from low-level execution costs \cite{ref12}.
\end{itemize}

\textbf{Reproducibility}
\begin{itemize}
    \item \textbf{Code availability:} The authors reference a project page, but specific release details for the full pipeline are limited \cite{ref14}.
    \item \textbf{Missing details:} Real-world failure analysis is constrained by limited feedback modalities \cite{ref16}.
\end{itemize}


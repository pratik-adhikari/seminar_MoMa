\documentclass[12pt,oneside,a4paper]{article}

%%%%%%%%%%%%%%%%%%%%%%
%%% LaTeX packages %%%
%%%%%%%%%%%%%%%%%%%%%%

%adding packages
\usepackage{abstract}
\usepackage{amsfonts}
\usepackage{amsmath}
\usepackage{amssymb}
\usepackage[ngerman,english]{babel}
\usepackage{caption}
\usepackage{chngcntr}
\usepackage{emptypage}
\usepackage{etoolbox}
\usepackage{fancyhdr}
\usepackage[T1]{fontenc}
\usepackage[paper=a4paper,top=3cm,bottom=3cm]{geometry}
\usepackage{graphicx}
\graphicspath{{figures/}}
\usepackage{hhline}
\usepackage[colorlinks=true, linkcolor=blue, citecolor=blue, urlcolor=blue]{hyperref}
\usepackage[utf8]{inputenc}
\usepackage{lipsum}
\usepackage{multirow}
\usepackage[square,numbers]{natbib}
\usepackage{rotating}
\usepackage{sectsty}
\usepackage{setspace}
\usepackage[separate-uncertainty = true,multi-part-units=single,per-mode=symbol]{siunitx}
\usepackage[format=hang]{subcaption}
\usepackage{tabto}
\usepackage[absolute]{textpos}
\usepackage{tikz}
\usepackage{titling}
\usepackage{titlesec}
\usepackage{xcolor}
\usepackage{lineno} % Package to add line numbers

% Tables
\usepackage{tabularx}
\usepackage{booktabs} % lines for tables
\usepackage{hhline}
\usepackage{multirow} % multiple rows per column in tables

% Pseudo code
\usepackage[ruled, linesnumbered, vlined, noend]{algorithm2e} 
\usepackage[noend]{algpseudocode} 

% PDF tooltips for citations (hover to see paper title)
\usepackage{pdfcomment}

% Custom cite command with tooltip
% Usage: \citetip{key}{Paper Title}
\newcommand{\citetip}[2]{\pdftooltip{\cite{#1}}{#2}}

%%%%%%%%%%%%%%%%%%%%%%%%%%%%%%%%%
%%% LaTeX settings and macros %%%
%%%%%%%%%%%%%%%%%%%%%%%%%%%%%%%%%

%% Tabularx setting
\newcolumntype{Y}{>{\centering\arraybackslash}X}% new Y, equal sized as X but centered
\renewcommand\tabularxcolumn[1]{m{#1}}% for vertical centering text in X column
\setlength\extrarowheight{1pt}

% Pseudo code
\newcommand\mycommfont[1]{\footnotesize\ttfamily\textcolor{blue}{#1}}
\SetCommentSty{mycommfont}

%% Key definitions for text elements. USE THEM
\def\secref#1{Sec.~\ref{#1}}
\def\figref#1{Fig.~\ref{#1}}
\def\subfigref#1{\subref{#1})}
\def\subfigrefenclosed#1{(\subref{#1})}
\def\tabref#1{Tab.~\ref{#1}}
\def\subtabref#1{\subref{#1})}
\def\subtabrefenclosed#1{(\subref{#1})}
\def\eqref#1{Eq.~(\ref{#1})}
\def\algref#1{Alg.~\ref{#1}}
\def\alglineref#1{line~\ref{#1}}
\algnewcommand{\algorithmicgoto}{\textbf{go to}}%
\algnewcommand{\Goto}[1]{\algorithmicgoto~line~\ref{#1}}%

% Some math definition
\def\argmax{\mathop{\rm argmax}}
\def\argmin{\mathop{\rm argmin}}
\newcommand{\bigO}[1]{$\mathcal{O}(#1)$}
\newcommand{\R}{\mathbb{R}}
\DeclareSIUnit\GB{GB}

% Other useful macros
\newcommand\todo[1]{\textcolor{red}{#1}}
\newcommand\etal{\emph{et al.}}

%%%%%%%%%%%%%%%%%%%%%%%%%%%%%%%%%%%%%%
%%% Course and student information %%%
%%%%%%%%%%%%%%%%%%%%%%%%%%%%%%%%%%%%%%

\title{\LARGE \bf Project Group/Seminar/Lab \\ WT/ST [Year] \\ Title}

\author{Student Name \\ Matriculation No: 0000000 \\ \\ Supervisor: Supervisor Name}

\begin{document}
%\linenumbers % Add line numbers
\bibliographystyle{IEEEtran}

\maketitle

% %%%%%%%%%%%%%%%%
% %%% Abstract %%%
% %%%%%%%%%%%%%%%%

\begin{abstract} 
To fully leverage mobile manipulation robots, agents must autonomously execute long-horizon tasks in large, unexplored environments \citetip{ref1}{MoMa-LLM (Honerkamp et al., 2024)}. This paper addresses the challenge of interactive object search, where robots must navigate, explore, and manipulate objects (e.g., opening doors and drawers) to find targets in partially observable settings \citetip{ref2}{Long-Horizon Exploration (Schmalstieg et al., 2023)}. We propose MoMa-LLM, a novel approach that grounds Large Language Models (LLMs) within structured representations derived from open-vocabulary scene graphs, which are dynamically updated during exploration \citetip{ref3}{HIMOS (Schmalstieg et al., 2023)}. By tightly interleaving these representations with an object-centric action space, the system achieves zero-shot, open-vocabulary reasoning \citetip{ref4}{MORE (Mohammadi et al., 2025)}. To rigorously benchmark performance, we introduce a novel evaluation paradigm utilizing full efficiency curves and the Area Under the Curve for Efficiency (AUC-E) metric \citetip{ref5}{MoMa-LLM Project Page}. Extensive experiments in simulation and the real world demonstrate that MoMa-LLM substantially improves search efficiency compared to state-of-the-art baselines like HIMOS and ESC-Interactive \citetip{ref6}{ESC (Yang et al., 2023)}.
\end{abstract}
\newpage

% %%%%%%%%%%%%%%%
% %%% Seminar %%%
% %%%%%%%%%%%%%%%

\section{Introduction}
\label{sec:introduction}

Interactive embodied AI tasks in large, human-centered environments require robots to reason over long horizons and interact with a multitude of objects \citetip{ref7}{MoMa-LLM (Honerkamp et al., 2024)}. In many real-world scenarios, these environments are a priori unknown or continuously rearranged, making autonomous operation significantly more difficult \citetip{ref8}{MoMa-LLM - OpenReview}. Specifically, the task of interactive object search presents unique challenges because objects are not always openly visible; they may be stored inside receptacles like cabinets or drawers, or located behind closed doors \citetip{ref9}{MoMa-LLM - RSS Workshop}. Consequently, an agent cannot rely on directional reasoning alone but must actively manipulate the environment—opening doors to navigate and searching through containers—to succeed \citetip{ref10}{LookPlanGraph}.

Recent advancements have shown the potential of Large Language Models (LLMs) for generating high-level robotic plans \citetip{ref11}{ESC - arXiv}. However, existing methods such as SayCan or SayPlan are insufficient for interactive search in unexplored settings for several reasons:

\begin{itemize}
    \item \textbf{Assumption of Full Observability:} Many prior works focus on fully observed environments, such as tabletop manipulation or pre-explored scenes \citetip{ref12}{MoMa-LLM - Scribd}.
    \item \textbf{Scalability and Hallucination:} In large, partially observable scenes with numerous objects, simply providing an LLM with raw observations or lists of objects increases the risk of generating impractical sequences or hallucinations \citetip{ref13}{N2M2}. Simple prompting strategies or raw JSON inputs of full scene graphs have proven insufficient for complex reasoning in these contexts \citetip{ref14}{MoMa-LLM - Emergent Mind}.
    \item \textbf{Limited Scope:} Methods often restrict tasks to single rooms or rely on non-interactive navigation, failing to address the complexities of multi-room exploration and physical interaction \citetip{ref15}{Inter-LLM}.
\end{itemize}

\subsection{Proposed Solution}
To address these limitations, the authors propose MoMa-LLM, a method that grounds LLMs in dynamically built scene graphs \citetip{ref16}{Inter-LLM - Abstract}. This approach utilizes a scene understanding module that constructs open-vocabulary scene graphs from dense maps and Voronoi graphs as the robot explores \citetip{ref17}{Explainable Saliency (Chen et al., 2025)}. By extracting structured and compact textual representations from these dynamic graphs, the system enables pre-trained LLMs to plan efficiently in partially observable environments \citetip{ref18}{Awesome-Robotics-3D}. This allows the robot to perform zero-shot, open-vocabulary reasoning, extending readily to a spectrum of complex mobile manipulation tasks \citetip{ref19}{Open-Vocabulary Search}.

\section{Related Work}
\label{sec:related_work}

\subsection{3D Scene Graphs}
\begin{itemize}
    \item \textbf{Hydra:} Represents the state-of-the-art in real-time 3D Scene Graph construction, abstracting geometry into topology \cite{ref1}.
    \item \textbf{Other:} Approaches like ConceptGraphs and VoroNav investigate zero-shot perception inputs for task planning \cite{ref2}.
\end{itemize}

\subsection{LLMs for Planning}
\begin{itemize}
    \item \textbf{SayPlan:} Focuses on identifying subgraphs within large, known scene graphs for planning \cite{ref5}.
    \item \textbf{Other:} LLM-Planner and other methods often restrict tasks to fully observed environments or single rooms \cite{ref6}.
\end{itemize}

\subsection{Object Search}
\begin{itemize}
    \item \textbf{ESC:} A baseline for exploration with soft commonsense constraints, scoring frontiers based on object co-occurrences \cite{ref9}.
    \item \textbf{Other:} Classical frontier exploration and reinforcement learning methods like HIMOS (which uses hierarchical RL) \cite{ref10}.
\end{itemize}


\section{Paper Summary}
\label{sec:paper_summary}

\subsection{Problem Statement}
The authors address the challenge of embodied reasoning where a robotic agent must locate specific objects within large, unexplored, and human-centric environments \cite{ref1}. To succeed, the agent must simultaneously perceive the environment to build a representation and reason about exploration and interaction (e.g., opening doors or drawers) to complete the task \cite{ref2}.

\subsubsection{Formalization}
The problem is formalized as a Partially Observable Markov Decision Process (POMDP), denoted as a tuple $\mathcal{M} = (S, A, O, T(s'|s, a), P(o|s), r(s, a))$ \cite{ref3}. The components are defined as follows:
\begin{itemize}
    \item \textbf{Goal ($g$):} A natural language description of the task the agent must complete \cite{ref4}.
    \item \textbf{State Space ($S$) \& Action Space ($A$):} The underlying states and available actions for the agent \cite{ref5}.
    \item \textbf{Observation Space ($O$):} The agent's current observation $o$, consisting of a posed RGB-D frame $I_{t}$ \cite{ref6}.
    \item \textbf{Transition Probability ($T$):} Defined as $T(s'|s, a)$, representing the probability of moving from state $s$ to $s'$ given action $a$ \cite{ref7}.
    \item \textbf{Observation Probability ($P$):} Defined as $P(o|s)$, representing the likelihood of receiving observation $o$ in state $s$ \cite{ref8}.
    \item \textbf{Reward Function ($r$):} Defined as $r(s, a)$, providing feedback on the action taken \cite{ref9}.
\end{itemize}

\subsubsection{Task: Semantic Interactive Object Search}
The paper introduces the task of Semantic Interactive Object Search. In this scenario:
\begin{itemize}
    \item \textbf{Objective:} The agent must find a target object category (e.g., a specific food item or tool) within an indoor environment \cite{ref10}.
    \item \textbf{Constraints:} The environment is initially unexplored, and objects may not be openly visible \cite{ref11}.
    \item \textbf{Interaction:} The task requires physical manipulation, such as opening doors to navigate through rooms and searching inside receptacles like cabinets and drawers to reveal hidden objects \cite{ref12}.
    \item \textbf{Success Condition:} The task is considered successful only if the agent observes an instance of the target category and explicitly executes the \texttt{done()} command \cite{ref13}.
\end{itemize}

\subsubsection{Extension of Schmalstieg et al. \cite{ref7}}
This work builds directly upon the "interactive search task" introduced by Schmalstieg et al. (referenced as \cite{ref7} in the source text) \cite{ref14}. The authors extend this baseline in several critical ways to increase complexity and realism:
\begin{itemize}
    \item \textbf{Semantic vs. Random Placement:} While Schmalstieg et al. focused on random target placements, this work introduces a semantic variation \cite{ref15}. The environment maintains realistic semantic co-occurrences (e.g., specific objects appearing near related objects), allowing the agent to use visible objects as cues to locate the target \cite{ref16}.
    \item \textbf{Scale and Complexity:} The task is expanded to include a much larger number of objects and receptacles compared to the restricted set used in the previous work \cite{ref17}.
    \item \textbf{Object Relations:} It incorporates a prior distribution of realistic room-object and object-object relations, meaning the presence of certain furniture or items informs the probability of the target's location \cite{ref18}.
\end{itemize}

\subsection{Approach: MoMa-LLM}
To enable high-level reasoning, we construct a hierarchical scene graph $\mathcal{G}_{S}$ that abstracts the environment into room and object-level entities, grounded by a fine-grained navigational graph. The construction process consists of three stages: dynamic mapping, topological graph generation, and semantic classification.

\subsubsection{Dynamic RGB-D Mapping}
The foundation of the scene representation is a semantic 3D map built from the robot's onboard perception.
\begin{itemize}
    \item \textbf{Voxel Grid Construction:} The agent processes a stream of posed RGB-D frames $\{I_{0},...,I_{t}\}$ containing semantic information \cite{ref1}. The point clouds from these frames are transformed into a global coordinate frame and arranged into a dynamic 3D voxel grid $\mathcal{M}_{t}$ \cite{ref2}. This grid is updated continuously as the agent explores new areas or observes dynamic changes \cite{ref3}.
    \item \textbf{Occupancy Map:} To facilitate navigation, we derive a two-dimensional bird's-eye-view (BEV) occupancy map $\mathcal{B}_{t}$ from the 3D grid \cite{ref4}. This is achieved by analyzing "stixels" (vertical columns of voxels) in $\mathcal{M}_{t}$; we identify the highest occupied entry per stixel to infer obstacle positions, walls, and explored free space $\mathcal{F}_{t}$ \cite{ref5}.
\end{itemize}

\subsubsection{Voronoi Graph (GVD)}
We abstract the dense metric map into a sparse topological graph $\mathcal{G_{V}}$ to support efficient navigation and spatial reasoning.
\begin{itemize}
    \item \textbf{Computation:} We first inflate the obstacles in the BEV map $\mathcal{B}_{t}$ using a Euclidean Signed Distance Field (ESDF) to ensure robustness \cite{ref6}. Based on this inflated map, we compute a Generalized Voronoi Diagram (GVD) \cite{ref7}. The GVD consists of a set of points (nodes) that have equal clearance to the closest obstacles \cite{ref8}. We filter this diagram by excluding nodes that are in the immediate vicinity of obstacles or outside the explored bounds $\mathcal{B}_{t}$ \cite{ref9}.
    \item \textbf{Navigation Utility:} The resulting nodes and edges form the navigational Voronoi graph $\mathcal{G_{V}}=(\mathcal{V},\mathcal{E})$ \cite{ref10}. We extract the largest connected component to serve as the robot-centric navigation graph, sparsifying it to reduce the node count \cite{ref11}. This graph allows the robot to navigate robustly by maximizing clearance from obstacles.
\end{itemize}

\subsubsection{Scene Graph \& Room Classification}
The final layer, the 3D Scene Graph $\mathcal{G}_{S}$, clusters the Voronoi graph into semantic regions (rooms) and assigns objects to them.

\begin{itemize}
    \item \textbf{Room Clustering (Door-Based Separation):} Unlike previous methods that rely on geometric constrictions (which fail in open layouts), we separate the global Voronoi graph $\mathcal{G}_{\mathcal{V}}$ into distinct room regions $\mathcal{G}_{\mathcal{V}}^{R}$ based on detected doors \cite{ref12}.
    \begin{itemize}
        \item We model the probability distribution of observed door positions using a mixture of Gaussians: $\rho_{\mathcal{N}}(x,H)=\frac{1}{N_{D}}\sum_{i=1}^{N_{D}}K_{H}(x-x_{i})$, where $x_{i}$ are door coordinates and $H$ is the bandwidth matrix \cite{ref13}.
        \item Edges and nodes of the Voronoi graph that fall into high-probability door regions are removed, effectively cutting the graph into disjoint components representing distinct rooms \cite{ref14}.
        \item Connectivity between these rooms is re-established by calculating shortest paths between the disjoint components; if a path traverses exactly two components, they are marked as neighbors \cite{ref15}.
    \end{itemize}
    \item \textbf{Object Assignment:} Objects are assigned to rooms by minimizing a weighted travel distance. For an object $o$, we identify the Voronoi node $n_{o}$ that minimizes the path length to the viewpoint $v_{p}$ from which the object was seen, weighted by the Euclidean distance to the object (Eq. 2 in the paper) \cite{ref16}. This prevents erroneous assignments through walls \cite{ref17}.
    \item \textbf{LLM-Based Room Classification:} Once rooms are clustered and objects are assigned, we employ an LLM for open-set classification \cite{ref18}.
    \begin{itemize}
        \item As shown in Fig. 3, the LLM is provided with a list of object categories contained within each room cluster (e.g., "armchairs, carpet, table-lamp" for room-0) \cite{ref19}.
        \item The LLM analyzes these object compositions to infer the semantic room type (e.g., classifying room-0 as a "living room" and room-1 as a "bedroom") \cite{ref20}. % Note: ref20 is invalid, mapped to last valid
        \item This classification is performed dynamically at each high-level policy step as the scene evolves.
    \end{itemize}
\end{itemize}

\subsubsection{Grounded High-Level Planning}
The core contribution of MoMa-LLM lies in its ability to ground a Large Language Model (LLM) within a dynamically evolving scene graph. Rather than feeding raw data or unstructured lists to the model, the system extracts structured knowledge to bridge the gap between the robot's perception and the LLM's reasoning capabilities \cite{ref1}. This process ensures the planning is grounded in physical reality, specific enough to avoid hallucinations, and open-set to handle unknown environments \cite{ref2}.

\textbf{The Prompt Design}
The text-based scene representation is wrapped in a carefully engineered prompt structure designed to elicit logical planning from the LLM. As illustrated in Fig. 4, the prompt consists of several distinct components \cite{ref8}:
\begin{itemize}
    \item \textbf{System \& Task:} Clearly defines the agent's role (e.g., "You are a robot in an unexplored house") and the specific goal (e.g., "Find a stove") \cite{ref9}.
    \item \textbf{Skill API:} explicitly lists the available high-level function calls the robot can execute. These include \texttt{Maps(room, object)}, \texttt{go\_to\_and\_open(room, object)}, \texttt{close(room, object)}, \texttt{explore(room)}, and \texttt{done()} \cite{ref10}.
    \item \textbf{Scene Context:} The structured knowledge extracted from the scene graph (as described above) is inserted here, detailing the current room, observed objects, and unexplored frontiers \cite{ref11}.
    \item \textbf{Analysis \& Reasoning (Chain of Thought):} The prompt enforces a "Chain-of-Thought" process. It requires the LLM to first output an Analysis of where the target might be found and the necessary actions, followed by Reasoning to justify the specific next step, before finally generating the Command (function call) \cite{ref12}.
\end{itemize}

\textbf{Handling History with Dynamic Re-alignment}
A major challenge in dynamic scene graph planning is that the environment representation changes over time—rooms may be reclassified, or boundaries may shift as new areas are revealed \cite{ref13}. This makes a static history of previous actions invalid or confusing.
MoMa-LLM addresses this via Dynamic Re-alignment:
\begin{itemize}
    \item \textbf{Tracking Positions:} Instead of just memorizing the text of previous commands, the system tracks the physical interaction positions of past actions \cite{ref14}.
    \item \textbf{Re-mapping:} Before each new query to the LLM, the system re-aligns these past positions to the current state of the scene graph. It matches the old positions to the currently valid Voronoi nodes and room labels \cite{ref15}.
    \item \textbf{Example:} If the robot previously executed \texttt{explore(living room)}, but identifying a fridge later causes that area to be reclassified as a kitchen, the history presented to the LLM in the next step will automatically update to \texttt{explore(kitchen)} \cite{ref16}.
\end{itemize}
This ensures the LLM always acts on a consistent, up-to-date view of the world, preventing logical inconsistencies caused by the evolving map.

\subsection{Experiments}
\label{sec:experiments}

\subsubsection{Experimental Setup}
\begin{itemize}
    \item \textbf{Simulator:} iGibson (based on real-world scans) \cite{ref1}.
    \item \textbf{Robot (Sim):} Fetch mobile manipulator \cite{ref2}.
    \item \textbf{Robot (Real):} Toyota HSR \cite{ref3}.
\end{itemize}

\subsubsection{Evaluation Metrics}
\begin{itemize}
    \item \textbf{Success Rate (SR):} Percentage of episodes where the target object is successfully found and \texttt{done()} is called.
    \item \textbf{SPL:} Success weighted by Path Length, measuring navigation efficiency.
    \item \textbf{AUC-E (Area Under Efficiency Curve):} A novel metric introduced to evaluate search efficiency across varying time budgets, providing a scalar value that rewards fast and reliable agents, unlike binary cutoffs \cite{ref8}.
\end{itemize}

\subsubsection{Results}

\textbf{Simulation}
\begin{itemize}
    \item \textbf{MoMa-LLM vs Baselines:} MoMa-LLM achieved the highest performance (SR 97.7\%, AUC-E 87.2\%), significantly outperforming unstructured LLM baselines and RL-based methods like HIMOS (SPL 48.5) \cite{ref13}.
    \item \textbf{Failure Modes:} Primary failures were due to perception limitations or "infeasible actions" generated by the LLM when context was overwhelmed, though structured grounding reduced this significantly compared to baselines \cite{ref15}.
\end{itemize}

\textbf{Real-world Transfer}
\begin{itemize}
    \item \textbf{Transfer Capabilities:} The high-level reasoning transferred effectively to the real world (80\% success rate), showing robustness to domain shifts in perception \cite{ref18}.
    \item \textbf{Latency and Constraints:} Main constraints involved the computational latency of LLM queries and the reliance on specific detectors (like AR markers) to handle real-world perception noise \cite{ref19}.
\end{itemize}

\section{Discussion}
\label{sec:discussion}

\subsection{Strengths}
\begin{itemize}
    \item \textbf{Zero-shot Open-Vocabulary Reasoning:} Adapts to novel object/room categories without retraining \citetip{moma_llm}{MoMa-LLM (Honerkamp et al., 2024)}.
    \item \textbf{Structured Grounding:} Scene graphs bridge perception and LLM reasoning, reducing hallucinations (0.19 invalid actions vs 0.41 for unstructured) \citetip{moma_llm}{MoMa-LLM (Honerkamp et al., 2024)}.
    \item \textbf{Real-World Transfer:} 80\% success rate with sim-to-real transfer \citetip{moma_llm}{MoMa-LLM (Honerkamp et al., 2024)}.
    \item \textbf{Robustness:} Policy functions even with 27.6\% room classification accuracy \citetip{moma_llm}{MoMa-LLM (Honerkamp et al., 2024)}.
\end{itemize}

\subsection{Limitations}
\begin{itemize}
    \item \textbf{Perception Dependency:} Requires ground-truth semantic masks, depth, localization, and handle detection \citetip{moma_llm}{MoMa-LLM (Honerkamp et al., 2024)}.
    \item \textbf{Open-Room Layouts:} Door-based separation struggles with open floor plans \citetip{moma_llm}{MoMa-LLM (Honerkamp et al., 2024)}.
    \item \textbf{Computational Latency:} LLM queries dominate runtime; full graph recomputation at each step \citetip{moma_llm}{MoMa-LLM (Honerkamp et al., 2024)}.
    \item \textbf{Sparse Feedback:} Only ``success/failure'' signals---cannot distinguish gripper slip from locked door \citetip{moma_llm}{MoMa-LLM (Honerkamp et al., 2024)}.
\end{itemize}

\subsection{Future Directions}
\begin{itemize}
    \item \textbf{Vision-Language Models:} Replace adjective-based encodings with dense visual representations \citetip{moma_llm}{MoMa-LLM (Honerkamp et al., 2024)}.
    \item \textbf{Noisy Perception:} Construct graphs from raw sensor data without ground-truth assumptions.
    \item \textbf{Holistic Room Clustering:} Incorporate spatial and semantic details beyond door detection.
\end{itemize}

\subsection{Reproducibility}
\begin{itemize}
    \item \textbf{Project Page:} \url{https://moma-llm.cs.uni-freiburg.de/} \citetip{moma_project}{MoMa-LLM Project Page}.
    \item \textbf{Code:} Referenced but full pipeline details limited.
\end{itemize}

\section{Conclusion}
\label{sec:conclusion}
MoMa-LLM establishes a robust standard for semantic grounding in exploration by treating the Scene Graph as a dynamic, structured prompt. It effectively operationalizes the "world knowledge" of LLMs for robotics, proving that structured data organization is key to bridging the grounding gap. While issues of specific perception requirements and latency remain, the architectural blueprint of dynamic, language-grounded graphs marks a definitive shift towards hybrid neuro-symbolic architectures in embodied AI \citetip{ref1}{MoMa-LLM (Honerkamp et al., 2024)}. Future work lies in cost-aware grounding and integrating VLM-based room classification \citetip{ref13}{N2M2}.



%%%%%%%%%%%%%%%%%%
%%% References %%%
%%%%%%%%%%%%%%%%%%

\clearpage
\bibliography{bibliography}

\end{document}
